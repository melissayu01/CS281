\begin{problem}[Hurdle Models for Count Data, 10pts]

  In this problem we consider predictive models of count data. For
  instance given information about the student $x$, can we predict how
  often they went to the gym that week $y$? A natural choice is to use a Poisson GLM
  i.e. $y$ conditioned on $x$ is modeled as a Poisson distribution.

  However, in practice, it is common for count data of this form to
  follow a bi-modal distribution over count data. For instance, our
  data may come from a survey asking students how often they went to
  the gym in the past week. Some would do so frequently, some would do
  it occasionally but not in the past week (a random zero), and a
  substantial percentage would never do so.

  When modeling this count data with generalized linear models, we
  may observe more zero examples than expected from our model.
  In the case of a Poisson, the mode of the distribution is the
  integer part of the mean. A Poisson GLM may therefore be inadequate
  when means can be relatively large but the mode of the output is 0. Such data is
  common when many data entries have 0 outputs and many also have much
  larger outputs, so the mode of output is 0 but the overall mean is
  not near 0. This problem is known as \textit{zero-inflation}.

  This problem considers handling zero-inflation with a two-part model called
  a \textit{hurdle model}. One part is a binary model such as a logistic model
  for whether the output is zero or positive. Conditional on a positive output,
  the ``hurdle is crossed'' and the second part uses a truncated model that
  modifies an ordinary distribution by conditioning on a positive
  output. This model can handle both zero inflation and zero
  deflation.

  Suppose that the first part of the process is governed by
  probabilities $p(y > 0\ |\ x) = \pi $ and $p(y = 0\ | \ x) = 1 - \pi$; and the second part
  depends on  $\{y\in \mathbb{Z} \ |\ y > 0\}$ and follows a probability mass
  function $f(y\ |\ \mathbf{x})$ that is truncated-at-zero. The
  complete distribution is therefore:
\begin{align*}
P(y = 0\ |\ x) & = 1- \pi\\
P(y = j\ |\ x) & = \pi \frac{f(j\ |\ \mathbf{x})}{1 - f(0\ | \ \mathbf{x})},\ j=1,2,...
\end{align*}
One choice of parameterization is to use a logistic regression model
for $\pi$: $$\pi = \sigma(\mathbf{x}^\top \mathbf{w}_1)$$ and use a Poisson GLM for $f$ with mean parameters $\lambda$ (see
Murphy 9.3): $$\lambda = \exp(\mathbf{x}^\top \mathbf{w}_2) $$

\begin{enumerate}[label=(\alph*)]
\item Suppose we observe $N$ data samples $\{(x_n, y_n)\}_{n=1}^N$.  Write down the log-likelihood for the hurdle model assuming an unspecified mass function $f$.
Give an maximum likelihood estimation approach for the specified parts of the model.

\item Assume now that we select Poisson distribution for $f$. Show
  that the truncated-at-zero Poisson distribution (as used in the
  hurdle model) is a member of the exponential family.  Give its the
  sufficient statistics, natural parameters and log-partition
  function.

\item What is the mean and variance of a truncated Poisson model with mean parameter $\lambda$? If we observe $n$ i.i.d. samples from a truncated Poisson distribution, what is the maximum likelihood estimate of $\lambda$? (Note: Give an equation which could be solved numerically to obtain the MLE. )

\item  Now assume that we using a hurdle model as a GLM with $f$ as a Poisson distribution. Show that
  this is a valid GLM (exponential family for $y$), derive its log-likelihood, and give its sufficient statistics.

\end{enumerate}
\end{problem}
